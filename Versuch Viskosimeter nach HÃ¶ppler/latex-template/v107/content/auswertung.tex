\section{Auswertung}
\label{sec:Auswertung}

Zuerst wurden die Dichten $\rho_{klein}$ und $\rho_{gross}$ der kleinen und großen Kugeln bestimmt. Dazu wurden Massen
$m_{klein} = 4.453\unit{\gram}$ und $m_{gross} = 4.953 \unit{\gram}$, sowie deren gemessener Durchmesser verwendet und
damit erst die Volumina mit $V = \frac{1}{6}\pi\left(d\right)^3$ berechnet:
\begin{align*}
  V_{klein} &= \frac{1}{6}\pi\left(1.55\unit{\centi\meter}\right)^3 \approx 1.9498\unit{\cubic\centi\meter}\\
  V_{gross} &= \frac{1}{6}\pi\left(1.58\unit{\centi\meter}\right)^3 \approx 2.0652\unit{\cubic\centi\meter}.
\end{align*}
Anschließend wurden mit der Formel: $\rho = \frac{m}{V}$ die Dichten der Kugeln berechnet:
\begin{align*}
  \rho_{klein} &= \frac{m_{klein}}{V_{klein}} \approx 2.2838\unit{\gram\per\cubic\centi\meter}\\
  \rho_{gross} &= \frac{m_{gross}}{V_{gross}} \approx 2.3983\unit{\gram\per\cubic\centi\meter}
\end{align*}