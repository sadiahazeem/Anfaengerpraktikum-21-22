\section{Diskussion}
\label{sec:Diskussion}
Bei der Auswertung zeigt sich das sich die Viskosität des Wassers wie nach der Andradeschen Gleichung zu erwarten war,
mit steigender Temperatur verringert. Doch zeigt sich in allen berechneten Werten ein fast gleichbleibende Abweichung
der dynamischen Viskosität von den Literaturwerten, da in \autoref{fig:plot} die Ausgleichsgerade trotzdem gut an den Messwerten anliegt.
Bei Raumtemperatur wurde $\eta$ bei der kleinen Kugel zu:
\begin{equation*}
    \eta = 1.32575\unit{\milli\pascal\second}
\end{equation*}
bestimmt, wobei der Literaturwert
\begin{equation*}
    \eta = 1.005\unit{\milli\pascal\second}
\end{equation*}
lautet.\\
Diese Abweichung aller Ergebnisse lassen sich auf systematische Fehler bei der Messung zurückführen.
Bei dieser wurde Beispielsweise die Zeit mit der Stoppuhr des Handys gemessen und der Zeitpunkt, an dem
gestoppt wird per Augenmaß gewählt. Dadurch ist die Zeitmessung an sich recht ungenau. Dazu kommt, dass die
Temperatur des Wassers in der Fallröhre in selbiger gemessen wurde, sondern bevor das warme Wasser durch pumpen
im umliegenden Zylinder verteilt wurde. Diese Verteilung ist ebenfalls nicht überall gleichmäßig und die Erwärmung des
Wassers in der Fallröhre auf die gewünschte Temperatur kann nicht garantiert sein. Zudem ist auf die Bestimmung der Apperatkonstante
der großen Kugel mithilfe der der kleinen ungenau, da diese explizit für die Kugeln bestimmt werden müssen. Außerdem können kleine Luftblasen
in der Fallröhre und an der großen Kugel nicht ausgeschlossen werden, da diese bereits eine Beschädigung und somit eine unebene Oberfläche vorwies.
Doch abgesehen von den systematischen Fehlern hat der Versuch die zu erwartenden Ergebnisse geliefert und durch die Reduktion dieser Fehler gut geeignet
um die Temperaturabhängigkeit der dynamischen Viskosität von Wasser zu messen.
