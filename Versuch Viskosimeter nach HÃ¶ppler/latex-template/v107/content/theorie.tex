\section{Theorie}
\label{sec:Theorie}

\subsection{Reynolds-Zahl}
\label{subsec:Reynolds-Zahl}

Die Reynolds-Zahl $Re$ gibt Auskunft darüber, ob eine Strömung laminar oder turbulent ist. \\
Das Verhalten von Objekten, wie einer Kugel, innerhalb dieser Strömung ist von verschiedenen Größen abhängig.\\
Darunter fallen die Dichte $\rho$, dynamische und kinematische Viskosität $\eta$ und $\nu$, 
und Strömungsgeschwindigkeit $v$ der Flüssigkeit 
und der Durchmesser $d$ der Kugel.

Die Reynolds-Zahl ist definiert über: 

\begin{equation}
    \label{eqn:Reynolds}
    Re = \frac{\rho v d }{\eta} = \frac{v d }{\nu}
\end{equation}

mit $\eta = \nu \rho$.

Ab einem kritischen Wert von $Re_{krit} \approx 2300$ wird eine bisher laminare Strömung bei 
minimalen Störungen turbulent, das heißt, es entwickeln sich Wirbel, und die Schichten gleicher Strömungsgeschwindigkeit, 
die bei laminarer Strömung parallel zueinander sind, vermischen sich miteinander.

Dieses Verhalten lässt sich durch die Kräfte $F_G$, $F_{Auftrieb}$ und $F_{Reibung}$, die auf die Kugel wirken, begründen.\\
Die Reibung bezieht sich hier auf die Flüssigkeitsschichten. \\

Für die Berechnung der Viskosität $\eta$ gibt es die Formel 

\begin{equation}
    \label{eqn:Eta}
    \eta = K (\rho_{Kugel} - \rho_{Fluessigkeit}) \cdot t
\end{equation}

K ist hier eine Apparaturkonstante, die sowohl durch die Kugelgeometrie, als auch die Fallhöhe bedingt wird.\\



Bei Betrachtung der Temperaturabhängigkeit der dynamischen Viskosität wird die Andradesche Gleichung genutzt:

\begin{equation}
    \label{eqn:Andrad}
    \eta(T) = A \cdot exp(\frac{B}{T})
\end{equation}

Dabei sind A und B Konstanten.

Bei der linearen Regression wird sie zu

\begin{equation}
    \label{eqn:AndradLin}
    ln(\eta(T)) = B \cdot \frac{1}{T} + ln(A)
\end{equation}
