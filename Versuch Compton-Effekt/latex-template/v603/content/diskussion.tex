\section{Diskussion}
\label{sec:Diskussion}

\subsection{Emissionsspektrum}
\label{subsec:disk_emission}

\begin{align*}
    E(K_{\alpha,exp}) = 8043 eV &&  E(K_{\alpha,lit}) = 8048,1 \\
    E(K_{\beta,exp}) = 8910 eV  &&  E(K_{\beta.lit}) = 8906,9
\end{align*}

Somit liegen die experimentiell bestimmten Werte mit einer prozentualen Abweichung von jeweils 0,1\% auffällig
genau an den Literaturwerten. \\
Dies bestätigt die Eignung des Versuchsaufbaus zur Bestimmung des Emissionsspektrums.\\
Da die Messung mit einem Röntgenapparat durchgeführt wird, welcher auch die Winkel des LiF-Kristalls einstellt, ist mit kleinen systematischen Fehlern zu rechnen.\\





\subsection{Compton-Wellenlänge}
\label{subsec:disk_compton}

\begin{align*}
    \lambda_{C, theo} = 2,42 \cdot 10^{-12}m && \lambda_{C, exp} = 3,8 \cdot 10^{-12}m \\
\end{align*}

Hier beläuft sich die prozentuale Abweichung auf den sehr hohen Wert von 54,9\%.\\
Eine solche Abweichung könnte auf einen Fehler in der Erhebung der Messwerte hindeuten, was jedoch nicht untersucht werden kann, 
da der Versuch nicht selbst durchgeführt wurde.\\
Der Compton-Effekt findet nicht im sichtbaren Spektrum statt, da die Zunahme der Wellenlänge
relativ zur Wellenlänge geringfügig ist.\\
Darum scheint die Streuung ohne Energieverlust zu passieren und es ist kein Compton-Effekt wahrzunehmen.\\
Bei Wellenlängen im sichtbaren Bereich würde die Wechselwirkung mit Elektronen zu andern Effekten führen.\\