\section{Diskussion}
\label{sec:Diskussion}

\subsection{Emissionsspektrum}
\label{subsec:disk_emission}

\begin{align*}
    E(K_{\alpha,exp}) = \SI{8043,3545}{\electronvolt} &&  E(K_{\alpha,lit}) = \SI{8047,78}{\electronvolt}\\
    E(K_{\beta,exp}) = \SI{8914,2038}{\electronvolt}  &&  E(K_{\beta.lit}) = \SI{8905,29}{\electronvolt}
\end{align*}

\noindent Somit liegen die experimentiell bestimmten Werte mit einer prozentualen Abweichung von jeweils 0,1\% auffällig
genau an den Literaturwerten. \\
Dies bestätigt die Eignung des Versuchsaufbaus zur Bestimmung des Emissionsspektrums.\\
Da die Messung mit einem Röntgenapparat durchgeführt wird, welcher auch die Winkel des LiF-Kristalls einstellt, ist mit kleinen systematischen Fehlern zu rechnen.\\





\subsection{Compton-Wellenlänge}
\label{subsec:disk_compton}

\begin{align*}
    \lambda_{C, theo} = \SI{2,43e-12}{\meter} && \lambda_{C, exp} = \left(3,2\pm 1,9\right)\cdot 10^{-12}\unit{\meter}\\
\end{align*}

\noindent Der theoretische Wert der Compton-Wellenlänge liegt im Fehlerintervall des experimentiell bestimmten. Daher ist die Messung
mit den gegebenen Unsicherheiten genau genug und geeignte die Compton-Wellenlänge zu ermitteln.\\
Der Compton-Effekt findet nicht im sichtbaren Spektrum statt, da die Zunahme der Wellenlänge
relativ zur Wellenlänge geringfügig ist.\\
Darum scheint die Streuung ohne Energieverlust zu passieren und es ist kein Compton-Effekt wahrzunehmen.\\
Bei Wellenlängen im sichtbaren Bereich würde die Wechselwirkung mit Elektronen zu andern Effekten führen.\\