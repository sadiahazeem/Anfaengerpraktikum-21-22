\section{Diskussion}
\label{sec:Diskussion}
Bei der ersten Methode die Zeitkonstante $RC$ über den Entladungsprozesses des Kondensators zu bestimmen fällt sofort im vergleich zu
den anderen bestimmten Werten von $RC$ auf, dass sich dieser um ein Zehnerpotenz von den anderen unterscheidet. Diese differenz kann darauf
zurückgeführt werden, dass bei der 1. Methode am Funktionsgenerator der TTL output und nicht wie dann bei allen anderen der $50\unit{\ohm}$
output verwendet wurde. Außerdem bringt das ablesen der Werte des Entladungsprozesses aus einem Bild eine zusätzliche unsicherheit mit sich.
Die in \autoref{abb:Entladung} erhaltene Kurve entspricht der theoretischen Erwartung eines exponentiellen abfalls nach \eqref{RelQ}.
Die Werte die sich für die Zeitkonstante bei den Methoden 2 und 3 ergeben liegen beide im selben Bereich, wobei bei Methode bei den Messwerten
in \autoref{fig:plot2} auffällt das diese vor allem für niedrigere Frequenzen immer mehr von der Originalfunktion abweichen, aber bei höheren dieser sehr nahe sind.\\
HIER NOCH ABWEICHUNG FÜR KLEINE F ERKLÄREN!!\\
Beim Plot für Methode 3 ist nur die Abweichung von der Originalfunktion und der Ausgleichsfunktion für große f auffällig was allerdings auf ein verstellen der TIME/DIV
für große f für eine bessere lesbarkeit auf dem Oszilloskopbildschirm zurückgeführt werden kann. Dieselbe Abweichung zeigt sich ebenfalls im Polarplot, wo dennoch alle
darin aufgetragen Winkel wie zu erwarten zwischen $0$ und $\frac{\pi}{2}$ liegen.
Ein vergleich der bestimmten Zeitkonstanten mit dem wahren Wert kann nicht durchgeführt werden, da die dafür erforderlichen größen $C$ und $R$ im Tiefpass nicht angegeben waren
oder gemessen werden konnten.\\
Integrator erklären!!