\section{Diskussion}
\label{sec:Diskussion}
Bei der ersten Methode, die Zeitkonstante $RC$ über den Entladungsprozesses des Kondensators zu bestimmen, fällt sofort im Vergleich zu
den anderen Messwerten zu $RC$ auf, dass sich dieser um eine Zehnerpotenz von den anderen unterscheidet. 

Diese Differenz kann darauf zurückgeführt werden, dass bei der ersten Methode am Funktionsgenerator der TTL-Output, und nicht, wie dann bei allen anderen, der $50\unit{\ohm}$-Output verwendet wurde. 

Außerdem bringt das Ablesen der Werte des Entladungsprozesses aus einem Bild eine zusätzliche Unsicherheit mit sich.

Die in \autoref{abb:Entladung} erhaltene Kurve entspricht der theoretischen Erwartung eines exponentiellen Abfalls nach \eqref{}.

Die Werte, die sich für die Zeitkonstante bei den Methoden 2 und 3 ergeben, liegen beide im selben Bereich, wobei bei Methode ?? bei den Messwerten
in \autoref{fig:plot2} auffällt, dass diese vor allem für niedrigere Frequenzen immer weiter von der Originalfunktion abweichen, aber bei höheren dieser sehr nahe sind.\\
HIER NOCH ABWEICHUNG FÜR KLEINE F ERKLÄREN!!\\
Beim Plot für Methode 3 ist nur die Abweichung von der Originalfunktion und der Ausgleichsfunktion für große Frequenzen auffällig, wie es allerdings auf ein Modulieren der TIME/DIV
für große Frequenzen zwecks besserer Lesbarkeit auf dem Oszilloskopbildschirm zurückgeführt werden kann. 

Dieselbe Abweichung zeigt sich ebenfalls im Polarplot, wo dennoch alle
darin aufgetragen Winkel wie zu erwarten zwischen $0$ und $\frac{\pi}{2}$ liegen.

Ein Vergleich der bestimmten Zeitkonstanten mit dem korrekten Wert kann nicht durchgeführt werden, da die dafür erforderlichen Größen Kapazität $C$ und Widerstand $R$ des Tiefpassfilters
 nicht auf dem Gerät angegeben waren oder gemessen werden konnten.\\

Die integrierende Eigenschaft des RC-Tiefpasses bei Frequenzen, bei denen $RC/T >> 1$ gilt, kommt daher, dass 
die über Widerstand und Kondensator abfallende Spannung einen Strom bedingt, der wiederum vom Kondensator gespeichert
beziehungsweise die Elektronen des Stroms summiert. Diese Elektronen sorgen selber für eine sich aufbauende Spannung
über den Kondensator, welche gleichzeitig auch die Ausgangsspannung bildet. So kommt es durch die Summation des
Stromes, der sich proportional zur Eingangsspannung verhält zu einer Integration der Eingangsspannung.