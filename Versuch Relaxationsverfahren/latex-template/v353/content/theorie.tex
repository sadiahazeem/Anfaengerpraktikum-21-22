\section{Theorie}
\label{sec:Theorie}

Die Relaxation eines Systems ist der Übergang aus dem ausgelenkten in den Ausgangszustand.
In diesem Versuch ist das zu untersuchende System ein RC-Schaltkreis.

Die Entladung des Kondensators durch einen Strom, der durch den Widerstand fließt, 
ist ein Beispiel für die Relaxation.

Zielsetzung des Versuchs ist, die Zeitkonstante des RC-Kreises zu ermitteln, sowie die
Messung der Amplitude der Kondensatorspannung und dem Phasenversatz von
Generator- und Kondensatorspannung.\\
Diese werden jeweils in Abhängigkeit von der Generatorfrequenz betrachtet.\\
Der letzte Teil des Versuches beschäftigt sich mit der Verwendung des RC-Gliedes als Integrator.

\subsection{Relaxationsgleichung}
\label{subsec:Die Relaxationsgleichung}

Die allgemeine Relaxationsgleichung besteht aus der beschränkten Größe A(t), der Proportionalitätsonstante c < 0 
und den konstanten
Werten A(0) und A($\infty$).

c variiert je nach Relaxationsvorgang und gibt Auskunft über die Geschwindigkeit des Entladeprozesses.

Die Änderungsrate von A wird als proportional zur Auslenkung angenommen:

\begin{equation} 
    \label{Rel_allg}
\frac{dA}{dt} = c[A(t)-A(\infty)]
\end{equation}

diese Gleichung wird dann mit $dt$ multipliziert und dann über das Intervall $[0; t]$ integriert. 
Es ergibt sich:

\begin{equation} 
    \label{Rel_RC}
    \textrm{ln}\frac{A(t)-A(\infty)}{A(0)-A(\infty)} = ct \\
\Longleftrightarrow\qquad A(t) = A(\infty) + [A(0)-A(\infty)] \cdot \textrm{exp}(ct)
\end{equation}


\subsection{Auf- und Entladevorgaenge am Kondensator}
\label{subsec:Entladevorgaenge am Kondensator}

Die klassische Elektrodynamik bringt die grundlegenden Beziehungen $I = \frac{U}{R}$ und $U_C = \frac{Q}{C}$ 
hervor.

Daraus lässt sich die Änderungsrate $\frac{dQ}{dt} = -\frac{1}{RC}Q(t)$ der Kondensatorladung bestimmen. 

Unter der Randbedingung, dass bei einem Entladeprozess $Q(\infty) = 0$ gelten muss, ergibt sich nach den gleichen
Umformungen, wie bei der allgemeinen Relaxationsgleichung ~\eqref{Rel_allg}

\begin{equation}
    \label{RelQ}
Q(t) = Q(0)\cdot \textrm{exp}(-\frac{t}{RC})
\end{equation}

$\frac{1}{RC}$ ist hier die Zeitkonstante (wie oben c).


\subsection{Auf- und Entladevorgaenge bei Wechselspannung}
\label{subsec:Entladevorgaenge bei Wechselspannung}

Die angelegte Wechselspannung $U(t) = U_0 \textrm{cos}(\omega t)$ lässt sich als periodische Anregung beschreiben, 
wie sie auch aus der Mechanik bekannt ist.

Zwischen $U_G$ und $U_C$ bildet sich der Phasenversatz $\varphi(\omega)$, sodass $U_C$ wie folgt beschrieben werden
kann:

\begin{equation}
    U_C = A(\omega) \cdot \textrm{cos}(\omega t + \varphi(\omega))
\end{equation}

Darus erhält man die Beziehung zwischen Frequenz und Phase:

\begin{equation}
    \label{PhiArctan}
    \varphi\left(\omega\right)=\arctan\left(-\omega RC\right).
\end{equation}

Womit sich auch die folgende Beziehung zwischen der Amplitude und der Kondensatorspannung herleiten:

\begin{equation}
    \label{eqn:bezAU}
    A\left(\omega\right) = \frac{U_0}{\sqrt{1+\omega^2R^2C^2}}.
\end{equation}

\subsection{Integration im RC-Kreis}
\label{subsec:Integration im RC-Kreis}

Bei geeigneten Frequenzen $\omega >> RC$ und Spannungen $|U_C| >> |U_G|$ erhält man durch die integrierende Funktion des RC-Gliedes 

\begin{equation}
    \label{integrator}
    U_C = \frac{1}{RC} \int_{0}^{t} U(\tau) d\tau
\end{equation}
