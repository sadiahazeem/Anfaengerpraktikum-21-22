\section{Theorie}
\label{sec:Theorie}

Die Relaxation eines Systems ist der Übergang aus dem ausgelenkten in den Ausgangszustand.
In diesem Versuch ist das zu untersuchende System ein RC-Schaltkreis.
Die Entladung des Kondensators durch einen Strom, der durch den Widerstand fließt, 
ist ein Beispiel für die Relaxation.

\subsection{Relaxationsgleichung}
\label{subsec:Die Relaxationsgleichung}

\subsection{Auf- und Entladevorgänge am Kondensator}
\label{subsec:Auf- und Entladevorgänge am Kondensator}

\subsection{Auf- und Entladevorgänge bei Wechselspannung}
\label{subsec:Auf- und Entladevorgänge bei Wechselspannung}

Die angelegte Wechselspannung $U(t) = U_0 cos(\omega t)$ lässt sich als periodische Anregung beschreiben, wie sie auch
aus der Mechanik bekannt ist.



\subsection{Integration im RC-Kreis}
\label{subsec:Integration im RC-Kreis}


\cite{sample}
