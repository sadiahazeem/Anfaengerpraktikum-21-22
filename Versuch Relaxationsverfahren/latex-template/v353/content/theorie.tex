\section{Theorie}
\label{sec:Theorie}

Die Relaxation eines Systems ist der Übergang aus dem ausgelenkten in den Ausgangszustand.
In diesem Versuch ist das zu untersuchende System ein RC-Schaltkreis.
Die Entladung des Kondensators durch einen Strom, der durch den Widerstand fließt, 
ist ein Beispiel für die Relaxation.

\subsection{Relaxationsgleichung}
\label{subsec:Die Relaxationsgleichung}

Die allgemeine Relaxationsgleichung besteht aus der beschränkten Größe A(t), der Proportionalitätsonstante c < 0 
und den konstanten
Werten A(0) und A($\infty$).
c variiert je nach Relaxationsvorgang und gibt Auskunft über die Geschwindigkeit des Entladeprozesses.

Die Änderungsrate von A wird als proportional zur Auslenkung angenommen:

\begin{equation} \label{eq:Relaxation}

\frac{dA}{dt} = c[A(t)-A(\infty)]
    
\end{equation}

diese Gleichung wird dann mit $dt$ multipliziert und dann über das Intervall $[0; t]$ integriert. 
Es ergibt sich:

\begin{equation}
    
ln\frac{A(t)-A(\infty)}{A(0)-A(\infty)} = ct \\

\Longleftrightarrow\qquad A(t) = A(\infty) + [A(0)-A(\infty)] \cdot exp(ct)

\end{equation}


\subsection{Auf- und Entladevorgänge am Kondensator}
\label{subsec:Entladevorgänge am Kondensator}

Die klassische Elektrodynamik bringt die grundlegenden Beziehungen $I = \frac{U}{R}$ und $U_C = \frac{Q}{C}$ 
hervor.
Daraus lässt sich die Änderungsrate $\frac{dQ}{dt} = -\frac{1}{RC}Q(t)$ der Kondensatorladung bestimmen. 
Unter der Randbedingung, dass bei einem Entladeprozess Q(\infty) = 0 gelten muss, ergibt sich nach den gleichen
Umformungen, wie bei der allgemeinen Relaxationsgleichung:

\begin{equation}
    
Q(t) = Q(0)\cdot exp(-\frac{t}{RC})

\end{equation}

$\frac{1}{RC}$ ist hier die Zeitkonstante (wie oben c).


\subsection{Auf- und Entladevorgänge bei Wechselspannung}
\label{subsec:Entladevorgänge bei Wechselspannung}

Die angelegte Wechselspannung $U(t) = U_0 cos(\omega t)$ lässt sich als periodische Anregung beschreiben, 
wie sie auch aus der Mechanik bekannt ist.

Zwischen $U_G$ und $U_C$ bildet sich der Phasenversatz $\varphi(\omega)$, sodass $U_C$ wie folgt beschrieben werden
kann:

\begin{equation}
    U_C = A(\omega) \cdot cos(\omega t + \varphi(\omega))
\end{equation}


\subsection{Integration im RC-Kreis}
\label{subsec:Integration im RC-Kreis}


\cite{sample}
