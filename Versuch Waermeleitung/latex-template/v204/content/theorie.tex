\section{Zielsetzung}
\label{sec:Zielsetzung}
Mit diesem Versuch soll die Wärmeleitfähigkeit von Aluminium, Edelstahl und Messing
mit statischer und dynamischer Temperaturänderung untersucht, und jeweils der Wärmeleitkoeffizient $\kappa$ bestimmt werden.
\section{Theorie}
\label{sec:Theorie}
Die Wärmeleitung eines Festkörpers ist ein Phänomen, das auftritt, wenn sich ein selbiger nicht im Temperaturgleichgewicht
befindet. \\
Dann findet ein Wärmeaustausch durch Phononen und freie Elektronen nach dem 2. Hauptsatz der Thermodynamik entlang des Temperaturgefälles statt.
Die Wärmemenge $dQ$, die in einer Zeit $dt$ in einem Stab der länge $L$, mit Dichte $\rho$, Querschnitt $A$ und spezifischer Wärme $c$ fließt, ist durch
\begin{equation}
    \label{eqn:WaermeMenge}
    dQ = -\kappa A\frac{\partial T}{\partial x}dt
\end{equation}
definiert.\\
Aus \eqref{eqn:WaermeMenge} erhält man durch Umformen die Gleichung für die Wärmestromdichte
\begin{equation}
    \label{eqn:WaermeDichte}
    \frac{1}{A}\frac{dQ}{dt} = j_w = -\kappa \frac{\partial T}{\partial x}.
\end{equation}
Dabei ist $\kappa$ der materialabhängige Wärmeleitkoeffizient des Stoffes.
Aus \eqref{eqn:WaermeMenge} und \eqref{eqn:WaermeDichte} ergibt sich die eindimensionale Wärmeleitgleichung
\begin{equation}
    \frac{\partial T}{\partial t} = \frac{\kappa}{\rho c}\frac{\partial^2 T}{\partial x^2},
\end{equation}
in welcher der Ausdruck $\frac{\kappa}{\rho c}$ als Temperaturleitfähigkeit bezeichnet wird. \\
Diese gibt an, mit welcher "Geschwindigkeit" sich ein Temperaturgefälle des entsprechenden Stoffes ausgleicht.
\\
Wird nun ein langer Stab periodisch erwärmt, entsteht eine Temperaturwelle der Form
\begin{equation}
    T\left(x,t\right) = T_{max}e^{-\sqrt{\frac{\omega \rho c}{2\kappa}}}\cos\left(\omega t - \sqrt{\frac{\omega \rho c}{2\kappa}}x\right).
\end{equation}
Für die Phasengeschwindigkeit $v$ dieser Temperaturwelle ergibt sich
\begin{equation}
    v = \frac{\omega}{k} = \sqrt{\frac{\omega 2 \kappa}{\rho c}}
\end{equation}
und für die Wärmeleitfähigkeit
\begin{equation}
    \label{eqn:kappa}
    \kappa = \frac{\rho c \left(\symup{\Delta}x\right)^2}{2\symup{\Delta}t\ln\left(\frac{A_{nah}}{A_{fern}}\right)}.
\end{equation}
Hierbei sind die Amplituden $A_{nah}$ und $A_{fern}$ gegeben durch Messung selbiger an den Messstellen $x_{nah}$ und $x_{fern}$.\\
Der Abstand dieser Messstellen ist $\symup{\Delta}x$ und $\symup{\Delta}t$ die Phasendifferenz der Welle zwischen den Messstellen.