\section{Diskussion}
\label{sec:Diskussion}

Zu Abweichungen von den zu erwartenden Werten kann es sowohl durch Randbedingungen, wie zum 
Beispiel die nicht ideale Isolierung, als auch durch den Einschwingvorgang bei der 
dynamischen Messung kommen. \\
Bei der Isolierung ist vor allem zu kritisieren, dass an den Enden der Stäbe gar kein 
isolierendes Material anliegt.\\

Darüber hinaus ist zu sagen, dass bei der Angström-Messung mit einer Periodendauer von 
200 Sekunden der Satz an Messdaten beim Speichern auf dem USB-Stick verloren gegangen ist, 
da das Messgerät abgestürzt ist.\\

Der Vergleich mit den Literaturdaten ----- ergibt Abweichungen, die ---

\begin{equation*}
    \kappa_{\textrm{Messing, gemessen}} =  \\
    %\kappa_{\textrm{Messing, Literatur} = 109 \frac{\textrm{W}}{\textrm{\mK}}
\end{equation*}

\begin{equation*}
    \kappa_{\textrm{Aluminium, gemessen}} =  \\
    %\kappa_{\textrm{Aluminium, Literatur} = 205 \frac{\textrm{W}}{\textrm{\mK}}
\end{equation*}

