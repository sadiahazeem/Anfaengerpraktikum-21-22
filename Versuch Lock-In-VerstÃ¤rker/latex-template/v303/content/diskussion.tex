\section{Diskussion}
\label{sec:Diskussion}
Es wird bestätigt, dass der Lock-In-Verstärker das Signal filtert, indem er die Spannung verstärkt. \\

Auffällig ist im ersten Versuchsteil, dass die Werte aus den Tabellen \ref{tab:DatenUnverrauscht} und 
\ref{tab:DatenVerrauscht} bzw Plots \ref{fig:unverrauscht} und \ref{fig:verrauscht} zu einer 
Sinus-, statt, wie erwartet, Cosinus-Funktion, passen. \\

Es lässt sich außerdem erkennen, dass die Werte durch das Verrauschen des Signals trotz Filterung ein wenig unregelmäßiger werden. \\
Im zweiten Versuchsteil ist zu kritisieren, dass der mehr oder weniger helle Hintergrund (Zimmerbeleuchtung an oder aus, draußen untergehende Sonne) bei der Messung großen Einfluss auf die Ergebnisse haben kann.\\
Dennoch lässt sich die $\frac{1}{r}$-Abhängigkeit in \autoref{fig:PhotoDet} gut erkennen. \\