\section{Diskussion}
\label{sec:Diskussion}



Der Fit an die Exponentialfunktion und die dazugehörigen Messwerte liegen nur leicht über der theoretisch 
errechneten Kurve, aber stellen doch ein zufriedenstellendes Ergebnis dar. \\




Der experimentiell bestimmte Wert von $R_{ap,exp} = 4,24 \: \mathrm{k\Omega}$ weicht um 25\% vom theoretisch berechneten Wert von 
$R_{ap,theo} = 5,72 \: \mathrm{k\Omega}$ ab. \\
Diese Abweichung lässt sich zum Teil durch die vernachlässigten Innenwiderstände des Aufbaus und Ungenauigkeiten 
beim Ablesen erklären.\\




In \autoref{fig:plotc2} zeigt sich ein Peak bei 26kHz. Dies ist offenbar die Resonanzfrequenz
des Schwingkreises. \\
Es fällt auf, dass die Theoriekurve deutlich über den gemessenen Werten liegen.\\ % wieso?

 

%\subsection{Bestimmung der Frequenzabhängigkeit des Phasenversatzes}
%\label{subsec:diskussion5d}