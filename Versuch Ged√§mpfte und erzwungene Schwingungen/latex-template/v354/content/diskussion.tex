\section{Diskussion}
\label{sec:Diskussion}

% theoriewerte k usw diskutieren

Der Fit an die Exponentialfunktion und die dazugehörigen Messwerte liegen nur leicht über der theoretisch 
errechneten Kurve, aber stellen doch ein zufriedenstellendes Ergebnis dar. \\




Der experimentiell bestimmte Wert von $R_{ap,exp} = 4,24 \: \mathrm{k\Omega}$ weicht um 25\% vom theoretisch berechneten Wert von 
$R_{ap,theo} = 5,72 \: \mathrm{k\Omega}$ ab. \\
Diese Abweichung liegt nicht im Fehlerbereich, lässt sich allerdings zum Teil durch die vernachlässigten Innenwiderstände des Aufbaus und Ungenauigkeiten 
beim Ablesen erklären.\\


Die Breite der Resonanzkurve liegt mit $f_+ - f_- = 6.5 \cdot 10^3 Hz$ sehr nahe am theoretischen Wert $f_+ - f_- = (6,43 \pm 0.005) \cdot 10^3 Hz$. 

In \autoref{fig:plotc2} zeigt sich ein Peak bei 26kHz. Dies ist offenbar die Resonanzfrequenz
des Schwingkreises. \\
Es fällt auf, dass die Theoriekurve deutlich über den gemessenen Werten liegen.\\ % wieso?

Beim Bestimmen der Werte, sowohl für die Kondensatorspannung, als auch zur Bestimmung des Phasenversatzes, stellt eine 
maßgebliche Fehlerquelle auch das Ablesen vom Dispaly des Oszilloskopes dar.\\ 

%\subsection{Bestimmung der Frequenzabhängigkeit des Phasenversatzes}
%\label{subsec:diskussion5d}