\section{Diskussion}
\label{sec:Diskussion}

%vgl mit literaturwert 


Es wird mit den Literaturwerten \cite{literaturwerte}
\begin{align*}
    E_{Lit, Messing} = 0,9 \: \textrm{bis} \: 1,0 \cdot 10^{11} N/m^2 && E_{Lit, Aluminium} = 0,7 \cdot 10^{11} N/m^2
\end{align*}
gearbeitet.

Für den Messingstab ergeben sich Abweichungen von 
\begin{align*}
    \frac{97,55 \cdot 10^{12} N/m^2 - 0,95 \cdot 10^{11} N/m^2}{0,95 \cdot 10^{11} N/m^2 * 100 } = 10,26
\end{align*}
mit dem ersten Messverfahren (einseitige Einspannung) und $E_{rund} = 97,55 \cdot 10^{12} N/m^2$.\\

Und mit der beidseitigen Auflage, also $E_{rund} = 123,43 \cdot 10^{12} N/m^2$, berechnet sich eine Abweichung von
\begin{align*}
    \frac{123,43 \cdot 10^{12} N/m^2 - 0,95 \cdot 10^{11} N/m^2}{0,95 \cdot 10^{11} N/m^2 * 100 } = 12,98
\end{align*}
vom Literaturwert.\\

Für den Aluminiumstab weicht der experimentiell ermittelte Wert des ersten Verfahrens $E_{eckig} = 109,7 \cdot 10^{12} N/m^2$ um
\begin{align*}
    \frac{109,7 \cdot 10^{12} N/m^2 - 0,7 \cdot 10^{11} N/m^2}{0,7 \cdot 10^{11} N/m^2 * 100 } = 15,66
\end{align*}
vom Literaturwert ab.\\

Einige Quellen für das Auftreten von Mesungenauigkeiten sind die Messuhren, die, teilweise 
schon ohne anhängendes Gewicht, ungleichmäßig verbogenen Stäbe und, wie sehr häufig, Ablesefehler.\\
Bei den Messuhren ist besonders auffällig, dass die linke Uhr, beim mittig anhängenden Gewicht, 
bei gleichen Abstand vom Mittelpunkt, beinahe doppelt so große Werte misst, wie die rechte Uhr.\\
Außerdem sind beide Messuhren sehr empfindlich für kleinste Erschütterungen und Deformationen der zu 
messenden Stäbe.\\
Hinzu kommt, dass wir beim Messen nicht bedacht haben, $D_0$ und $D_m$ (mit anhängendem Gewicht)
an den gleiche x-Stellen zu messen, sodass in der graphischen Auswertung mit der Interpolation 
bzw. Regression gearbeitet werden muss, um $D(x) = D_m - D_0$ zu bestimmen.