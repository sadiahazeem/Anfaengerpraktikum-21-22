\section{Diskussion}
\label{sec:Diskussion}

Die experimentiell bestimmten Elastizitätsmodule der beiden Stäbe ergeben sich zu --- für die 
einseitige Einspannung und --- für die beidseitige Auflage. \\
Entsprechende Literaturwerte ergeben sich nach --- zu ---.\\

Einige Quellen für das Auftreten von Mesungenauigkeiten sind die Messuhren, die, teilweise 
schon ohne anhängendes Gewicht, ungleichmäßig verbogenen Stäbe und, wie sehr häufig, Ablesefehler.\\
Bei den Messuhren ist besonders auffällig, dass die linke Uhr, beim mittig anhängenden Gewicht, 
bei gleichen Abstand vom Mittelpunkt, beinahe doppelt (?) so große Werte misst, wie die rechte Uhr.\\
Außerdem sind beide Messuhren sehr empfindlich für kleinste Erschütterungen und Deformationen der zu 
messenden Stäbe.\\
Hinzu kommt, dass wir beim Messen nicht bedacht haben, $D_0$ und $D_m$ (mit anhängendem Gewicht)
an den gleiche x-Stellen zu messen, sodass in der graphischen Auswertung mit der Interpolation 
bzw. Regression gearbeitet werden muss, um $D(x) = D_m - D_0$ zu bestimmen.