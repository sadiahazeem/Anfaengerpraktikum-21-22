\section{Auswertung}
\label{sec:Auswertung}

Die Eckdaten des Geräts lauten 
\begin{align*}
  R_1 = 67,2 \pm 0,1 \Omega \\
  R_2 = 682 \pm 0,5 \Omega \\
  L = (16,87 \pm 0,05) \cdot 10^{-3}H \\
  C = (2,060 \pm 0,003) \cdot 10^{-9}F.   % ------- 10^-9 oder doch 10^-3? es passt besser -9 weil sonst R_ap komisch wäre 
\end{align*}






\subsection{Berechnung von Abklingdauer und des eff. Dämpfungswiderstandes}
\label{Abklingdauer und R_daempf}

Die Abklingdauer $T$ und der effektive Dämpfungswiderstand $R_{eff}$ können anhand der Zeitabhängigkeit der Kondensatorspannungsamplitude 
bestimmt werden.\\

Der theoretische Verlauf der Kurve lässt sich mit $u_0 = 4V$ durch 
\begin{equation*}
    u(t) = 4 \cdot e^{-\frac{R}{2 \cdot L \cdot t}} = 4 \cdot e^{- 1,99 \cdot 10^{-3} \cdot t}
\end{equation*}
berechnen.\\
Hier wird der Widerstand $R_1$ genutzt. 

\begin{figure}[H]
  \centering
  \includegraphics[scale=0.07]{content/v354a2.png}
  \caption{Die gemessene Kondensatorspannung im RLC-Kreis bei anliegender Rechteckspannung. Zeit t in $20\mu s/Div$ und U in $0.2V/Div$.}
  \label{fig:MessApp}
\end{figure}



%\begin{figure}
%  \centering
%  \includegraphics{build/plot_abklingdauer.pdf}
%  \caption{Der Abklingvorgang, theoretisch und experimentiell.}
%  \label{fig:plot_abklingdauer}
%\end{figure}





\subsection{Bestimmung des Widerstandes beim aperiodischen Grenzfall}
\label{R_ap}

Der experimentiell bestimmte Wert für $R_{ap}$, also der Widerstand, ab dem kein Überschwingen mehr passiert, beträgt $R_{ap,exp} = 4,24 \: k\Omega$. \\
Ein entsprechender Theoriewert dazu lässt sich mit 
\begin{equation*}
  R_{ap,theo} = \sqrt{\frac{4L}{C}} = \sqrt{\frac{4 \cdot 16,87 \cdot 10^{-3}H}{2,060 \cdot 10^{-9}F}} = 5,72 \: k\Omega
\end{equation*}
ermitteln.






\subsection{Bestimmung der Frequenzabhängigkeit der Kondensatorspannung}
\label{Kondensatorspannung}

\begin{table}[H]
  \centering
  \caption{Die Kondensatorspannung $U_C$ in Abhängigkeit der Frequenz der Erregerspannung.}
  \begin{tabular}{cc}
    \toprule
    {$f \mathbin{/} \unit{\kilo\hertz}$} &
    {$U_C(t) \mathbin{/} \unit{\volt}$} \\
    \midrule
    20 & 0.75 \\
    22 & 1.05 \\
    24 & 1.35 \\
    26 & 1.65 \\
    28 & 1.50 \\
    30 & 1.10 \\
    32 & 0.80 \\
    34 & 0.55 \\
    36 & 0.50 \\
    38 & 0.40 \\
    40 & 0.35 \\
    42 & 0.30 \\
    44 & 0.25 \\
    46 & 0.20 \\
    48 & 0.16 \\
    50 & 0.19 \\
    52 & 0.15 \\
    54 & 0.13 \\
    56 & 0.12 \\
    58 & 0.11 \\
    60 & 0.10 \\

    \bottomrule
  \end{tabular}
  \label{tab:Tabelle1}
\end{table}


% plot 1

\begin{figure}
  \centering
  \includegraphics{build/plotc.pdf}
  \caption{Die Frequenzabhängigkeit der Kondensatorspannung $U_C$ halblogarithmisch gegen die Frequenz $f$ aufgetragen.}
  \label{fig:plotc}
\end{figure}



% plot 2

\begin{figure}
  \centering
  \includegraphics{build/plotc2.pdf}
  \caption{Die Frequenzabhängigkeit der Kondensatorspannung $U_C$ linear gegen die Frequenz $f$ aufgetragen.}
  \label{fig:plotc2}
\end{figure}




\subsection{Bestimmung der Frequenzabhängigkeit des Phasenversatzes zwischen $U_{err}$ und $U_{a}$}
\label{Phasenversatz}


\begin{table}[H]
  \centering
  \caption{Der Phasenversatz $\varphi$ in Abhängigkeit der Frequenz der Erregerspannung.}
  \begin{tabular}{cccc}
    \toprule
    {$f \mathbin{/} \unit{\kilo\hertz}$} &
    {$a \mathbin{/} \unit{\micro\second}$} &
    {$b \mathbin{/} \unit{\micro\second}$} &
    {$\varphi \mathbin{/} rad$} \\
    \midrule
    
    20 & 1.0 & 22.0 & 0.286 \\
    22 & 1.5 & 20.0 & 0.471 \\
    24 & 2.5 & 18.0 & 0.872 \\
    26 & 3.5 & 16.5 & 1.332 \\
    28 & 4.5 & 15.0 & 1.884 \\
    30 & 5.0 & 14.0 & 2.244 \\
    32 & 5.5 & 14.0 & 2.468 \\
    34 & 5.5 & 17.5 & 1.975 \\
    36 & 6.0 & 12.0 & 3.142 \\
    38 & 5.0 & 11.0 & 2.856 \\
    40 & 5.0 & 11.0 & 2.856 \\
    42 & 4.5 & 10.5 & 2.693 \\
    44 & 5.0 & 10.0 & 3.142 \\
    46 & 4.0 & 9.0  &  2.793 \\
    48 & 4.0 & 9.0  &  2.793 \\
    50 & 4.5 & 9.5  &  2.976 \\
    52 & 4.0 & 8.0  &  3.142 \\
    54 & 3.5 & 7.5  &  2.932 \\
    56 & 3.5 & 7.5  &  2.932 \\
    58 & 3.5 & 7.5  &  2.932 \\
    60 & 3.0 & 7.0  &  2.693 \\
    
    \bottomrule
  \end{tabular}
  \label{tab:Tabelle2}
\end{table}



\begin{figure}
      \centering
      \includegraphics{build/plotd.pdf}
      \caption{Die Phase zwischen $U_C$ und $U_{err}$ in Abhängigkeit der Frequenz von $U_{err}$, halblogarithmisch dargestellt.}
      \label{fig:plotd}
    \end{figure}



    \begin{figure}
      \centering
      \includegraphics{build/plotd2.pdf}
      \caption{Die Phase zwischen $U_C$ und $U_{err}$ in Abhängigkeit der Frequenz von $U_{err}$, linear dargestellt.}
      \label{fig:plotd2}
    \end{figure}



    Die Frequenzabhängigkeiten der Kondensatorspannung und Phasenverschiebung ergeben sich 
    aus der Frequenzabhängigkeit der komplexen Widerstände, aus denen sich die Gesamtimpedanz ergibt. \\
    \begin{equation*}
      |Z_{gesamt}| = \sqrt{R^2+ (\omega L - \frac{1}{\omega C})^2}
    \end{equation*}

    Die Kurve der Impedanz lässt sich wie folgt darstellen.

    \begin{figure}
      \centering
      \includegraphics{build/plotZ.pdf}
      \caption{Die Impedanz $Z$ gegen die Frequenz der Spannung aufgetragen.}
      \label{fig:plotZ}
    \end{figure}