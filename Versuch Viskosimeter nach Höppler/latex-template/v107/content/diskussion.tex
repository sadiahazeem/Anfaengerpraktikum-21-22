\section{Diskussion}
\label{sec:Diskussion}
Bei der Auswertung zeigt sich, dass sich die Viskosität des Wassers, wie nach der Andradeschen Gleichung zu erwarten war,
mit steigender Temperatur verringert. 

Auch tritt in allen berechneten Werten ein fast gleichbleibende Abweichung
der dynamischen Viskosität von den Literaturwerten auf, 
da in \autoref{fig:plot} die Ausgleichsgerade trotzdem gut an den Messwerten anliegt.

Bei Raumtemperatur wurde $\eta$ bei der kleinen Kugel zu:
\begin{equation*}
    \eta = 1.32575\unit{\milli\pascal\second}
\end{equation*}
bestimmt, wobei der Literaturwert
\begin{equation*}
    \eta = 1.005\unit{\milli\pascal\second}
\end{equation*}
lautet.\\

Diese regelmäßige Abweichung aller Ergebnisse lassen sich auf systematische Fehler bei der Messung zurückführen.

Dort wurden beispielsweise die Zeiten mit der Stoppuhr des Handys gemessen und der Zeitpunkt, an dem
gestoppt wird, per Augenmaß gewählt. 

Dadurch ist die Zeitmessung an sich relativ ungenau. 

Hinzu kommt, dass die Temperatur des Wassers in der Fallröhre nicht in selbiger 
gemessen wurde, sondern bevor es durch Pumpen
im umliegenden Zylinder verteilt wurde. 

Diese Verteilung ist ebenfalls nicht überall gleichmäßig und die Erwärmung des
Wassers in der Fallröhre auf die gewünschte Temperatur kann nicht garantiert sein. 

Zudem ist die Bestimmung der Apparaturkonstante
der großen Kugel, mit Hilfe der Konstante der kleinen Kugel, sehr ungenau, da sie 
explizit für die Kugeln bestimmt werden müsste. 

Außerdem können kleine Luftblasen
in der Fallröhre und an der großen Kugel nicht ausgeschlossen werden, 
da diese bereits eine Beschädigung in Form einer unebenen Oberfläche aufwies.

Abgesehen von den systematischen Fehlern hat der Versuch die zu erwartenden Ergebnisse geliefert 
und ist bei Reduktion dieser Fehler gut geeignet, 
um die Temperaturabhängigkeit der dynamischen Viskosität von Wasser zu messen.
